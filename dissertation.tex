\documentclass[pdftex,12pt,a4paper]{report}

% Some useful packages.
\usepackage{amsmath}
\usepackage{siunitx}
\usepackage{graphicx}
\usepackage{verbatim}
\usepackage{mhchem}
\usepackage{textcomp}
\usepackage{setspace}

% Reduces margins substantially.
\usepackage{geometry}
\newgeometry{margin=2.5cm}

% Allows headers and footers.
\usepackage{fancyhdr}
\pagestyle{fancy}
% Get rid of annoying line under header.
\renewcommand{\headrulewidth}{0pt}

% Properly fixed by installing texlive-fonts-recommended
% Temporary fix for missing font problem.
% N.B. I know this is wrong, but there is a problem with missing fonts otherwise.
% \renewcommand{\textdegree}{$^{\circ}$ }

\lhead{}
\chead{}
\rhead{}

\newcommand{\ts}{\textsuperscript}
\newcommand{\HRule}{\rule{\linewidth}{0.5mm}}

% Harvard style references.
\usepackage[backend=biber,style=authoryear,sorting=nyt,dashed=false]{biblatex}
\renewcommand*{\nameyeardelim}{\addcomma\space}
\addbibresource{references/references.bib} % note the .bib is required

% 12,000 words max.

% overall structure:
% Abstract
% Intro
% Data Sources
% Development of the Hurricane Detection Procedure
% Conclusion
% Appendices
% References

% Common typos:
% possibly

% TODOs: 29

% TODO: 2
% TODOCITE: 1
% TODOWRITE: 9
% TODOREF: 2
% TODOFIGREF: 2
% TODOFIG: 2
% TODOCHECK: 1
% TODOIMPROVE: 10

\title{Objective Tracking and Classification of Hurricanes in the 20\ts{th} Century Reanalysis Dataset}
\author{Mark Muetzelfeldt - UCL Department of Geography}

\date{29 August, 2014}

\begin{document}

\begin{titlepage}

\begin{center}

\textsc{\LARGE University College London}\\[1.5cm]

\textsc{\Large MSc Environmental Modelling Dissertation}\\[0.5cm]

% Title
\HRule \\[0.4cm]
{ \LARGE \bfseries Objective Tracking and Classification of Hurricanes in the 20\ts{th} Century Reanalysis Dataset \\[0.4cm] }

\HRule \\[1.5cm]

% Author and supervisor
\begin{minipage}{0.4\textwidth}
\begin{flushleft} \large
\emph{Author:}\\
Mark \textsc{Muetzelfeldt}
\end{flushleft}
\end{minipage}
\begin{minipage}{0.4\textwidth}
\begin{flushright} \large
\emph{Supervisor:} \\
Dr.~Chris \textsc{Brierley}
\end{flushright}
\end{minipage}
\\[0.5cm]
29\ts{th} August, 2014
\\[0.5cm]

This research dissertation is submitted for the MSc Environmental Modelling at University College London


\vfill
% Bottom of the page

\end{center}

\end{titlepage}


% Magic incantation to make a blank page with no page number that does not mess up page ordering.
\clearpage
\thispagestyle{empty}
\null
\addtocounter{page}{-1}%
\newpage

\onehalfspacing
\section*{Abstract}

% Should be <= 400 words.
% Not included in word count.
% ~262 words so far.
A procedure for the detection of hurricanes in the 20\ts{th} Century Reanalysis Project dataset is
developed. Tracks are derived from the dataset by tracking vorticity maxima over time. These derived
tracks are matched against best tracks taken from the International Best Track Archive for Climate
Stewardship (IBTrACS) best tracks dataset. Six different tracking configurations are evaluated:
using the vorticity at the 0.995 times the surface and \SI{850}{hPa} pressure levels and using the
original resolution data, data downscaled by cubic interpolation to two and three times the original
resolution. It is found that using downscaling to three times the resolution provides a better match
between the derived tracks and the best tracks. Tracks at the different pressure levels produce
similar results, and the \SI{850}{hPa} pressure level is chosen on theoretical grounds.

The derived tracks are used to obtain extra fields from the 20\ts{th} Century Reanalysis
Project dataset, such as the local minimum pressure and the temperature at the \SI{850}{hPa}
pressure level. These extra fields, along with the information obtained by matching the derived
tracks to the best tracks, are used to produce training and validation datasets for supervised
machine learning classification algorithms. Different classification algorithms are evaluated based
on classification success and error metrics. It is found that a Stochastic Gradient Descent
classifier combined with a simple threshold classifier produces the highest combined score on these
metrics.

The hurricane detection procedure is run over the course of the 20\ts{th} Century, and a comparison
between the estimated number of hurricanes and the number as recorded in the IBTrACS dataset is
performed. It is found that TODOWRITE

\begin{center}
\textbf{Word count:} 8194 % 7229 + 965 (captions) 24/08/14 18:14
\end{center}

\section*{Acknowledgements}

I would like to thank Dr. Chris Brierley for his guidance and assistance in this dissertation.
Technical expertise from Qinling Wu proved useful in setting the direction for this project.
Talking to Joshua Studholme helped to provide context for the field and ideas for analysis, as well
as giving me some pointers on extratropical transitions. % TODOIMPROVE: CLUMSY
Dr. Kevin Hodges (Reading University) provided many useful suggestions for the tracking algorithm.
Finally, numerous discussions with Robert Muetzelfeldt were invaluable for keeping me on track and
for sounding out various ideas.

\newpage

\tableofcontents

% Everything from here to auto-critique is included in word count.
\chapter{Introduction}
% From the dissertation handbook:
% Introduction, presenting the research problem, rationale, context and outline objectives,
% aims/objectives (possibly as a formal hypothesis).

% Rationale.
% ----------

% Impacts of hurricanes.
Hurricanes are one of nature's most impressive and devastating extreme weather events. The costliest
hurricane this century, hurricane Katrina, was responsible for 1833 deaths and caused \$108
billion USD in damage \parencite{knabb2006tropical}. % TODOWRITE: Need another sentence here.
Hurricanes are therefore of great importance socially,
economically and, in hurricane Katrina's case, even politically \parencite{kellner2007katrina}.

% Heat transport by hurricanes.
Hurricanes, and more generally tropical cyclones, are also known to be a major contributor to the
poleward heat flux that evens out the temperatures between the poles and the tropics, carrying an
estimated \SI{1.4e15}{W} towards the poles \parencite{emanuelContribution2001}. This
contribution is large enough that it may affect the thermohaline circulation
\parencite{hu2009effect}, a major component of the coupled atmosphere-ocean system of the Earth. So,
coupled with their socio-economic impact, understanding hurricanes is of intrinsic scientific
interest too, helping to understand the Earth's climate and to produce more accurate General
Circulation Models.

% Effects of global warming on frequency/intensity.
From a theoretical perspective, increasing global mean temperatures are predicted to lead to an
increase in the intensity of hurricanes \parencite{emanuel1987dependence}. This is backed up by
computer modelling, which predicts a 6\% increase in maximum wind speed under conditions of a rise
in global temperatures over the next 80 years \parencite{knutson2004impact}. Given that a
hurricane's destructive potential varies as the maximum surface wind speed cubed
\parencite{emanuel2005increasing}, this would lead to a 20\% increase in hurricanes' destructive
potential. In line with the theory, this destructive potential is found to have risen ``markedly''
since the 1970s \parencite{emanuel2005increasing}. As well as increases in the intensity of
hurricanes, increases in their frequency have been detected \parencite{goldenberg2001recent,
webster2005changes, holland2007heightened}.

% Trends in hurricanes over time leading on to...
Most studies into hurricane frequency and intensity tend to rely on best tracks databases as their
primary data source, such as the IBTrACS database used in this study
\parencite{knappInternational2010}, or the tropical cyclone database from the National Hurricane
Center \parencite[HURDAT;][]{jarvinen1984tropical}. These databases suffer from some problems when
going into the middle and early 20\ts{th} century, such as the lack of satellite data before the
1970s, and the lack of reliable aircraft reconnaissance data before the mid 1940s
\parencite{chang2007number}. Accordingly, in this study, the approach taken is a different one.
Instead of using the numbers compiled by various agencies as the primary source of information into
hurricanes, the primary data will be taken from reanalysis data, specifically the 20\ts{th} Century
Reanalysis (20CR) Project \parencite{compoTwentieth2011}. Hurricanes will be detected in these
data, and the detection procedure will be trained and validated on recent data (from 1990 to 2009),
where reliable best tracks exist. Once the detection procedure has been trained, it can be used to
run over the whole of the 20\ts{th} century, providing an objective measure of the number of
hurricanes in each year.

\section{Research Problem}
% Detection of hurricanes in 20CR dataset.
% * Reason: spans 20th C.
% * List other reanalyses that do not.
% * Problems with this dataset: low resolution, despite trying to make it objective, it will likely
% change over time, making trend detection tricky (laying groundwork for talking about this in
% discussion).

% Mention NCEP/NCAR Reanalysis 1
% Mention ERA-40
% Mention JRA-55

% BRIEF description of what reanalysis is,
A global reanalysis project is an attempt to reconstruct the complete historical state of the
atmosphere at a given spatial and temporal resolution, assimilating an incomplete series of
measurements of atmospheric variables (a more complete description is given in Section
\ref{sec:20crp}). % TODOIMPROVE: They are not incomplete, just sparse in time/space.
Several reanalysis projects exist, such as NCEP/NCAR Reanalysis \parencite{kalnay1996ncep,
kistler2001ncep}, ECMWF ERA-40 \parencite{uppala2005era} and the JRA-55
\parencite{ebita2011japanese}. These reanalysis only go back to 1948, 1957 and 1958  respectively,
and so would not be useful for looking into hurricanes at the start of the 20\ts{th} century. The
20CR on the other hand covers over 140 years, from 1871 to 2012, and is therefore suitable for
studying the number of hurricanes across the entire 20\ts{th} century. It contains an estimate of
the state of the atmosphere on a T62 (2\textdegree) spatial resolution and at 6-hourly timesteps,
with variables such as the longitudinal and latitudinal wind speeds at the 0.995 times the surface
pressure level, or the Near Surface Pressure Level (NSPL), and the temperature at the \SI{850}{hPa}
pressure level recorded.
% TODOWRITE: say that I'm going to be detecting hurricanes etc. i.e. spell out what makes it
% suitable.

\section{Existing Tracking and Detection Algorithms}
% Context.
% --------

% Tracking/detection: Literature review. Lots and lots of references.
% Tracking: why it's a good idea to track then detect. Links to Hodges 1 and 2.
% Detection: GCM:
% Talk about role of resolution
% Talk about
% Discuss different approaches.
% Detection: Reanalyses.
% List some previous attempts to detect hurr/tropical cyclones in reanalyses.
% Talk about the primary difference: the existence of best tracks datasets.

% TODOIMPROVE: Needs a bit of re-writing due to not being on the output of a GCM for reanalyses.
Several reasons exist for wanting to detect hurricanes in the output of GCMs. Often the goal is to
look into the effects of climate change with a warming climate \parencite{bengtsson1996will,
sugi2002influence, yoshimura2006influence}. When running this sort of analysis, there is no objective measure of whether or not
an individual detection of a hurricane is correct or not, although often tests are done to make sure
that the resulting climatologies are realistic. This is in contrast to running the detection using
the output of a GCM that is part of a historical reanalysis dataset, where it is possible to match
the detected hurricane to a given tracks in a given best track database, as was done by
\textcite{walsh1997objective}.

% TODOIMPROVE: Ditto, make it clear that it's acting on global atm. output, not nec. GCM output.
There are two closely related approaches to detecting hurricanes in the output of GCMs. The first is
to examine the output of the GCM at every timestep, looking for a cell or cells which meet certain
criteria, and then produce tracks based on combining cells that meet these in criteria in different
timesteps. These criteria are normally involve vorticity at the \SI{850}{hPa} pressure level,
maximum wind speed, and temperature anomalies at different pressure levels. Additionally, the tracks
produced have to be longer than a certain time duration, typically 1.5 to 2 days. Examples of
studies that take this approach are \textcite{bengtsson1995hurricane, walsh1997objective}.

The second approach of detecting hurricanes is to first track some feature from one timestep to the
next (typically vorticity maxima), and then apply a set of criteria to each point in these tracks to
see whether they represent hurricanes or not. Normally, a time duration of the tracks must be longer
than a certain duration as well, again typically 1.5 days. These tracks can yield extra information
about the hurricanes, such as where their were formed, which is important in deriving details about
cyclogenesis \parencite{marchok2002ncep}. Useful information about the fate of the cyclone can also
be found, such as whether it undergoes an extratropical transition \parencite{hart2001climatology,
studholme2014objective}. This is the approach taken by \textcite{hodges1994general,
hodges1999adaptive, camargo2002improving}, and is also the approach that will be used in this study.

Both these approaches rely on thresholds being set for the criteria in advance of running the
detection procedure, although there may be some sensitivity analysis done on these thresholds as in
\textcite{walsh1997objective}. These thresholds can be somewhat arbitrary and can vary with resolution
\parencite{walsh2007objectively}. Two studies that have looked at T63 resolution (very close to the T62 used by
the 20CR dataset), have used values of \SI{6e-5}{s^{-1}} as a threshold value for hurricanes in the
\SI{850}{hPa} vorticity \parencite{bengtsson2006storm, bengtsson2007may}. Therefore any value used in this
study should be of a similar magnitude to this number. The thresholds can also vary from basin to
basin \parencite{camargo2002improving}, and thus need to be chosen quite carefully.

This study takes a different approach to choosing these thresholds. Instead of deciding what they
should be before running the analysis, the analysis is run, and the values of e.g. vorticity and
temperature at the \SI{850}{hPa} pressure level are recorded for each point along each track. Due to
the matching of the tracks derived from the 20CR data and the best tracks, it is then possible to
see which values of these fields correspond to hurricanes, and which do not. This allows for setting
of the thresholds \textit{after} the bulk of the processing has been done, thus greatly speeding up
the process of finding optimum values for these thresholds. It also permits the use of machine
learning classification algorithms, which it is hoped will improve the accuracy of the hurricane
detection, as measured by the number of true and false positives. This is a similar approach to
\textcite{walsh1997objective}, who analysed the numbers of false and missing tropical cyclones over September
1989 and October 1992, and results of this study will be compared against these. It bears repeating
that this approach is only possible because this study is running the hurricane detection over a
historical reanalysis dataset, otherwise this would not be possible, as there is no objective
standard by which to judge success.

% Objectives/Project Structure.
% -----------------------------
% Outline of objectives for this project.
% Finish with a run-down of the structure of the rest of the dissertation, including why it deviates
% from the traditional Intro/Methods/Results/Discussion/Conclusion structure.

% TODOIMPROVE: think about moving last objective to the top, and explaining that development of the
% procedure involves these steps.
The objectives of this study are to develop a tracking algorithm that is capable of producing tracks
derived from the 20CR dataset that can be successfully matched to the tracks in the IBTrACS best
tracks database. Following from this, these derived tracks will be used to collect extra information for
each of the points in the track, such as temperature at \SI{850}{hPa}. This information will then
used, along with the information about whether a particular point corresponds with a hurricane or
not, to train different classification algorithms to correctly predict whether each point is a
hurricane or not. The classification algorithm which performs best on success and error metrics will
be chosen, and this process will form the hurricane detection procedure. This will then be run
over the course of the 20\ts{th} century, and aggregate statistics about the number of hurricanes
over this duration will be presented, along with a comparison between the estimated number of
hurricanes from the 20CR dataset and the IBTrACS database.

The following chapter will present more detail on the two data sources used in this study: the 20CR
dataset and the IBTrACS database. Chapter \ref{chap:hurricane_detection_proc} will then detail the
steps in the detection procedure, presenting a logical progression of the development along with
results pertinent to the following sections of the development. Chapter \ref{chap:results_analysis}
will then look into the results of running the procedure over the course of the 20\ts{th} century.
Chapter \ref{chap:discussion} will then discuss these results in the wider context of hurricane
detection and trends over the 20\ts{th} century, and Chapter \ref{chap:conclusion} will provide a
summary and conclusion of the findings. This deviates slightly from the traditional thesis structure
due to the time spent working on the detection procedure, and the desire to present this development
in a logical manner where the subsequent sections depend on the results of the previous sections.

\chapter{Data Sources}

\section{20\ts{th} Century Reanalysis Project}
\label{sec:20crp}

% Key points:
% * Global reanalysis 1871-2012
% * Uses only SST, surf press, sea ice
% * Relatively bias free because of above (compared to say ERA-40)
% * spatio-temporal resolution
% * includes uncertainty
% * 56 ensemble members
The 20\ts{th} Century Reanalysis Project (20CR) is a project whose aim is to produce a consistent
best estimate of the state of the atmosphere by including information from a variety of different
sources using data assimilation \parencite{compoTwentieth2011}. Sea Surface Temperature (SST) and
sea ice data from the Hadley Centre Sea Ice and SST dataset \parencite{rayner2003global} are used to drive
a GCM - the NCEP Global Forecast System \parencite{kanamitsu1989description, kanamitsu1991recent}.
This is combined with pressure observations from the International Surface Pressure Databank
\parencite{yin2008international}, which are assimilated with the forecasts provided by the GCM by
means of an Ensemble Kalman Filter (EKF). It provides estimates at a 2\textdegree\ spatial
resolution, and on a 6-hour timestep, from 1871 to 2012 (more years will be added using the same
process in due time). A level of consistency is ensured by using the same process from the beginning
of the reanalysis to the end, and by using only SST, sea ice and pressure data. This avoids the
problems seen in other reanalysis projects, such as ERA-40 \parencite{uppala2005era}, which
incorporate as much data as possible. This biases more recent years, as e.g. satellite data from the
1970s onwards provides far more observations than exist before this period. Due to the use of an
EKF, it is also possible for the 20CR to provide estimates for the uncertainty of each of its
estimates at each timestep, although this information is not used as part of this study.

% necessity of analysing single EM (ref compo2011)
The GCM and EKF are run with 56 ensemble members, meaning that for every timestep there are 56
estimates for the state of the atmosphere. According to \textcite{compoTwentieth2011}, individual
tracking of storms is best done through tracking them in individual ensemble members, instead of
looking at the ensemble mean. This is because the ensemble mean, in the presence of only a few
observations, exhibits less synoptic variability. % TODOIMPROVE: unclear, explain better.
Therefore the individual ensemble members were used in this study. Consequently, the analysis
carried out in this study had to be repeated 56 times, and this provides the classification
algorithms used in Section \ref{sec:classification} far more samples than would otherwise have been
the case.

\subsection{Technical Details}
% How it works:

% Runs a GCM - NCEP GFS - to produce background fields
% Technical details on GCM
The GCM used by the reanalysis was the NCEP Global Forecast System
\parencite{kanamitsu1989description, kanamitsu1991recent}, with
parameterisation as specified by \textcite{saha2006ncep}. The version used in the reanalysis was
slightly modified to permit time varying \ce{CO2} and volcanic aerosols. These, along with incoming
radiation, were used as specified in \textcite{saha2010ncep}. The model was run at a horizontal
resolution of T62 and a vertical resolution of 28 vertical hybrid sigma-pressure levels as described
by \textcite{juang2005discrete}. This horizontal resolution is between the T42 and T106 resolutions
analysed in \textcite{bengtsson1995hurricane}, and therefore should be capable of producing
hurricane-type vortices, as the T42 model in that study was found to be capable of producing
vortices, although it was noted that these vortices were of lower intensity and that their overall
structure was less realistic than in the T106 GCM.

% Technical details EKF
The EKF is similar to the Monte Carlo approximation proposed by \textcite{evensen1994sequential,
evensen2003ensemble}. It is 'deterministic', and is based on the 'Ensemble Square Root Filter' of
\textcite{whitaker2002ensemble}.
% TODOWRITE: say more about EKF.

% Assimilates the observations and the output from the GCM using an EKF to provide an estimate as to
% the state of the atmosphere at each timestep
The process by which estimates are produced involves three main steps:

\begin{enumerate}
    \item Spin up of GCM
    \item Assimilate pressure data using the EKF (repeated)
    \item Use GCM to predict the state of the atmosphere at the next timestep (repeated)
\end{enumerate}

56 individual GCMs are initialised with climatalogical states taken from 14 months before the start
of the production period, i.e. the period over which estimates are produced. The GCMs are then run
for 14 months, to ``reduce the effect of the initial condition in the lower layers of the land
model.'' This makes use of the SST and sea ice data. At this point, the 56 different states of the
atmosphere from the GCMs are used as background states for the EKF, which assimilates the pressure
observations for the current timestep, to produce 56 estimates for the state of the atmosphere based
on the GCM output and the observations. These estimates are then used as the starting point for the
GCMs, which predict what the state of the atmosphere will be in 6 hours time, and this is used as
the background state for the EKF, which assimilates the pressure observations at the new timestep.

% Feeds this estimate back into the GCM to run another forecast
% Consistency of each EM not an issue (ref compo2011)
This continual predict and assimilate process repeats for 5 years, and the whole process is repeated
over the full course of the reanalysis period. It should be noted that this 5 year period leads to a
discontinuity in the data: every 5 years on the 1\ts{st} of January starting from 1876, the
individual ensemble members will not be continuous. This is not important for this study, because
there are very few hurricanes in the Atlantic over the December-January divide. However, it would be
more important if one were looking at tropical cyclones over the Pacific, when this is an active
period.

\subsection{International Surface Pressure Databank}
The pressure data comes from the International Surface Pressure Databank version 2 (ISPDv2)
\parencite{yin2008international}. This takes its data from over 40 international agencies, and
combines them into one usable databank. One particularly notable data source is the IBTrACS dataset
used in this study (see Section \ref{sec:ibtracs}). This is of interest because this study treats
these two data sources as being independent, although this is not the case. However, due to the data
assimilation process, and all the other factors that go into making the reanalysis data such as the
GCM and every other source of pressure data, it is thought that the effect of this data being
assimilated should not overly affect the conclusions of this study. Indeed, in Section
\ref{sec:extra_field_collection}, where a comparison is made between a derived track and a best
track, the surprising aspect is the weak correlation of the resulting pressures, which suggests
that the IBTrACS best track pressure data is not solely responsible for the pressure along the
track's course.

% Results summary and usage for hurricane detection:
% * compo2011
% * neff2013
% * emanuel2010

\subsection{Representations of Hurricanes in the 20CR dataset}

In \textcite{neff2013analysis}, the path of the Galveston Hurricane of the 8\ts{th} September, 1900
is analysed. The analysis finds that the hurricane is detectable through a minimum in the pressure
field, even though the mean pressure field was used. They find that the track derived from the 20CR
for this hurricane was systematically to the north-east of the corresponding best track (taken from
the IBTrACS dataset). However, this study points to the fact that even hurricanes as far back as the
beginning of the 20\ts{th} century are well represented. Figure \ref{fig:galveston} shows the
pressure and vorticity fields for one ensemble member and one date of this hurricane. In this
figure, the 2\textdegree\ resolution of the data is clearly visible, as is the vorticity maximum and
the pressure minimum.

\begin{figure}[ht!]
    \centering
    \includegraphics[width=\textwidth]{figures/galveston_1900-9-7_18-00_em0}
    \caption{Vorticity and pressure fields over the Gulf of Mexico from September the 7\ts{th}, 1900
        at 1800 UTC (taken from ensemble member 1). Shows the hurricane that would go on to hit Galveston,
        Texas. The hurricane is clearly visible in both the vorticity and pressure fields.}
    \label{fig:galveston}
\end{figure}

% TODO not sure this should be here.
In a more comprehensive study, \textcite{emanuel2010tropical} used an earlier version of the 20CR
dataset to look at tropical cyclone activity from 1908-1958. He used dynamical downscaling (i.e.
using a high resolution model with the 20CR data providing boundary conditions for this model, not
to be confused with the simple cubic spline interpolation downscaling used in this study) to better
represent the structure of the tropical cyclones. He found a ``marginally significant increase'' in
the number of northern hemisphere tropical cyclones, along with a high correlation between the power
dissipation index from derived from the 20CR reanalysis and the best tracks from the same period.

\section{IBTrACS Best Tracks Dataset}
\label{sec:ibtracs}
The International Best Track Archive for Climate Stewardship (IBTrACS)
\parencite{knappInternational2010} aims to be a homogenised global collection of best tracks of
tropical storms and tropical cyclones. It collects best tracks from numerous Regional Specialized
Meteorological Centers (RSMCs) and Tropical Cyclone Warning Centres (TCWCs), and combines these
together, removing duplicates and trying detect common discrepancies, such as dates being different
by one day and the positions between tracks not being the same.

The dataset contains data for each track at 6 hourly time intervals (taken at 0000, 0600, 1200, 1800
UTC daily). It includes data on position, wind speed and pressure. It is split into separate basins,
and in this study only the North Atlantic (NA) basin will be utilised. The primary source of data
for this basin is the Atlantic Hurricane Database \parencite[HURDAT;][]{jarvinen1984tropical}, indeed
there are ``no new storms in the NA or EP basins'' in the IBTrACS dataset.

% TODO Explain what a hurricane timestep is.
% TODO Add eras to hurricane timesteps. 

\begin{figure}[hm!]
    \centering
    \includegraphics[width=\textwidth]{figures/hurr_per_year}
    \caption{Hurricane timesteps from 1890 to 2009.}
    \label{fig:hurr_per_year}
\end{figure}

\begin{figure}[hm!]
    \centering
    \includegraphics[width=\textwidth]{figures/yearly_hurr_dist}
    \caption{Hurricane distribution over the course of the year. The 1\ts{th} of June and the
    1\ts{th} of December are shown, these indicate the start and end of the analysis period in this
study. It can be seen that the vast majority of hurricanes fall within this period.}
    \label{fig:yearly_hurr_dist}
\end{figure}

% TODOWRITE:
% Talk about usage for hurricane trending over 20th C, and problems with this.
% Go into more detail generally.

\chapter{Hurricane Detection Procedure}
\label{chap:hurricane_detection_proc}

% Overall description to go here, with flow chart.
% Mention:
% * project URLs
% * choice of tracking then classification vs classifying individual CDPs
% * amount of processing required

The hurricane detection procedure forms the basis of the rest of this study. It is a multi-step
procedure which logically follows the flowchart shown in Figure \ref{fig:hurricane_detection_proc}.
The procedure consists of separate logical steps, starting from the data in the 20CR and IBTrACS
datasets, and proceeding through a series of steps until detected hurricanes are found. 

\begin{figure}[ht!]
    \centering
    \includegraphics[width=0.7\textwidth]{figures/hurricane_detection_procedure}
    \caption{Flowchart showing the steps in the hurricane detection procedure. Processes are shown
    by diamonds, and data or results are shown by rectangles. }
    \label{fig:hurricane_detection_proc}
\end{figure}

\newpage
The computer code developed to perform these processes was all written in Python, apart from two
helper methods for calculating vorticity as in Section \ref{sec:vort}, which were written in C for
computational efficiency. The code for this project is all available at the following location:
https://github.com/markmuetz/stormtracks/. The code follows the structure set out in the flowchart
closely, there are e.g. Python modules called c20data, tracking and matching, although the flowchart
represents a simplification of the code. The Python project also contains infrastructure code for
allowing concurrent processing on the UCL computers, where the bulk of the processing was done,
following a simple master/slave pattern, with one computer designated as the master, farming out
jobs to multiple slave computers that would carry out the processing, whether it be e.g. matching or
extra field collection. 

There were constraints on the available disk space on the UCL computers, so the
processing was batched into decades, with the 20CR data for one decade downloaded, the processing
run, and the results gathered, before deleting that decades data and moving onto the next one.
Bearing in mind one decade's worth of 20CR data takes up about \SI{112}{GB} of disk space,
downloading all of this data and processing it took about 12 hours, even with the processing spread
across multiple computers. Therefore the complete run over the period from 1890 to 2009 took around
six days, notwithstanding the various failures and problems hit along the way. Given that the
processing was run on 12 computers, and accounted for roughly half of the time spent for one decade,
this represents around 36 days of processing time.

\section{Data Processing}

\subsection{Wind Fields and Vorticity}
\label{sec:vort}

In the 20CR data, the longitudinal and latitudinal wind fields, $u$ and $v$ respectively, were used
to calculate the vorticity and wind speeds. These are available at three different pressure levels:
Near Surface Pressure Level (NSPL), \SI{850}{hPa} and \SI{250}{hPa}. The \SI{250}{hPa} fields were
only used to determine whether or not they were suitable for tracking of hurricanes, and it was
found that they did not produce suitable tracks, and were not further used (see Section
\ref{sec:results_tracking}). The formula for vorticity, $\omega$, is given by:

\begin{equation}
    \omega = \frac{\partial v}{\partial x} - \frac{\partial u}{\partial y}
    \label{eqn:vorticity}
\end{equation}

The surface of the earth, $R_e$, is taken as its mean radius of \SI{6371}{km}, and $\phi$, $\lambda$
denote the latitude and longitude coordinates in radians. $u_{i, j}$ denotes the longitudinal wind
speed at grid cell $i, j$. Then $\omega$ can approximated on the surface of the Earth using a
2\ts{nd} order approximation: % TODOIMPROVE: CLUMSY

% Calculation in code is a little different because it uses degrees instead of radians:
% More like vorticity = du / (2 * dlon * cos(lat * pi / 180) * Earth_circ / 360) - dv / ((2 * dlat) * Earth_circ / 360)

\begin{equation}
    \omega \approx \frac{\Delta v}{2 \Delta x} - \frac{\Delta u}{2 \Delta y} = \frac{v_{i,j+1} - v_{i,j-1}}{2 R_e \cos{\phi} \Delta \lambda} - \frac{u_{i+1,j} - u_{i-1,j}}{2 R_e \Delta \phi }
    \label{eqn:vorticity_2nd_order}
\end{equation}

The vorticities at the NSPL and \SI{850}{hPa} levels were calculated.

\subsection{Maxima and Minima detection}
\label{sec:methods_maxima_minima}

% TODO: Define PSL higher up?
For the vorticity fields, maxima are of primary interest as these represent grid cells where the
wind is rotating most strongly. Conversely, for the Pressure at Sea Level (PSL) field, minima are of
primary interest. A maximum or minimum was defined as a cell whose value was greater or less than
all eight of its surrounding cells. Vorticity maxima play a vital role in deriving tracks from the
20CR data, and pressure minima are useful both for comparison with the best tracks data and
classification of the tracks.

\subsection{Downscaling of Data}

% TODOCITE See if I can find a direct citation for this. Interpolation is mentioned in hodges1994,
% but not using a spline.
The vorticity field was downscaled to a finer resolution. This was done so as when tracks were
derived from these data, the so called ``staircase effect'' \parencite{hodges1994general}, whereby
tracks are seen to follow the grid cells at the resolution of the 20CR data, was minimised (an
example of this ``staircase effect'' can be seen in Figure \ref{fig:katrina_individual_match_em7}).
This downscaling was accomplished using a cubic spline interpolation, and the data were downscaled
to two and three times their original resolutions. The efficacy of this downscaling in improving the
position of the vorticity maxima will be demonstrated in Section \ref{sec:results_tracking}.

\subsection{Data Processing Results}

Figure \ref{fig:katrina_data_proc} shows the progression of the data from its raw form, to a derived
vorticity, and then a downscaled to three times the original resolution vorticity field. Figure
\ref{fig:katrina_max_mins} shows an example vorticity and pressure field taken from around Hurricane
Katrina, and the corresponding maxima for the vorticity and minima for the pressure fields.
TODOWRITE:
say interesting things about these figures.

\begin{figure}[hb!]
    \centering
    \includegraphics[width=0.9\textwidth]{figures/katrina_data_proc}
    \caption{Three different views of Hurricane Katrina over the Gulf of Mexico as it appears in the
        20CR on the 27\ts{th} August, 2005 at 1800 UTC (taken from ensemble member 1). The left
        figure shows the raw 20CR wind fields with contours fitted based on the wind speed. The
        middle figure shows the corresponding vorticity field, with the 2\textdegree\ resolution of
        the data clearly visible. The right figure shows the data after they have been downscaled to
        three times the original resolution, showing the more precise location of the vorticity
        maximum.}
    \label{fig:katrina_data_proc}
\end{figure}

\begin{figure}[hb!]
    \centering
    \includegraphics[width=0.9\textwidth]{figures/katrina_max_mins}
    \caption{Vorticity and pressure maximum and minimum detection as shown by Hurricane Katrina,
        taken on the 27\ts{th} August, 2005 at 1800 UTC (taken from ensemble member 1). Maxima are
        shown by red dots, and minima by black crosses. The large number of vorticity maxima should
        be noted, which is why a vorticity cutoff as described in Section \ref{sec:tracking} is
        employed. A clear vorticity maximum and pressure minimum can be seen indicating the position of
        Hurricane Katrina in this figure.}
    \label{fig:katrina_max_mins}
\end{figure}

\section{Tracking}
\label{sec:tracking}

From initial experimentation, it was found that tracking on minima of the pressure field did not
produce suitable tracks for identifying storm systems. This was due to the pressure field being
subject to synoptic scale disturbances that mean that minima may not be present due to the
surrounding gradient of this field. Whilst it is possible to apply high pass filtering to remove
these disturbances, it was decided to follow \textcite{reed1988evaluation, thorncroft2001african}
and produce derived tracks from the 20CR data based on vorticity maxima.

% TODOIMPROVE: This section is confusing on re-reading it. I think I need to make it clearer that I'm going to
% compare different configurations.
For each timestep, the vorticity maxima were calculated, as per Section
\ref{sec:methods_maxima_minima}. This was done for both the NSPL and \SI{850}{hPa} pressure
levels, and for the one, two and three times the actual 20CR resolution, yielding a total of six
possible configurations for the tracking. These various configurations were compared, and the
results can be seen in Section \ref{sec:results_tracking}. Maxima that were below a cutoff value
of \SI{2.5e-5}{s^{-1}} were not used as part of the tracking algorithm, as these represent weak maxima
that are unlikely to be part of even tropical storms. This is similar to the ``relaxed threshold
criteria'' used in \textcite{camargo2002improving}. This choice is revisited in Section
TODOREF where some further justification is given for this value.

The tracking algorithm used was a simple nearest neighbour tracking algorithm, with a modification
that counted two or more vorticity maxima that were sufficiently close as being part of the same storm
system, with the location and strength of the maximum given by the strongest of
these maxima. This cutoff distance was set as eight times the distance between the cells at
0\textdegree\ N, or approximately \SI{1800}{km}. This distance was found to reduce problems of
vorticity maxima merging and splitting. The algorithm worked as follows:

%  TODOIMPROVE: flowchart?
\begin{enumerate}
    \item Detect all vorticity maxima for the first timestep. If two maxima are less than the cutoff
        distance apart, combine both maxima, taking the strength and position of the strongest of
        the two maxima
    \item Make each of the vorticity maxima in the first timestep the beginning of a derived track
        (whose length could end up being only one)
    \item Detect all vorticity maxima in the second timestep, combining close weaker maxima into
        stronger. % TODOIMPROVE: not totally clear that I'm talking about the combining step here.
    \item Calculate nearest neighbour in the second timestep to first timestep, and set this as the
        next member of the derived track
    \item Move onto the third timestep and repeat the process
\end{enumerate}

Whilst this algorithm is considerably less sophisticated that that of \textcite{hodges1994general},
it was found to produce sufficiently good tracks to pick up most of the best tracks in the IBTrACS
dataset, and almost all of the best tracks that were hurricanes (Section
\ref{sec:results_tracking}). A discussion of other tracking algorithms can be found in Section
\ref{sec:discussion_tracking_algs}.

\subsection{Tracking Results}
\label{sec:results_tracking}

Once tracks have been derived from the 20CR, it is possible to match them to the corresponding best
tracks. The corresponding best track must be close the derived track in space, and its temporal
duration must overlap the duration of the derived track. In this project, an overlap of 6 timesteps,
or one and a half days, was required for there to be a match between a derived track and a best
track. Additionally, the mean distance from the derived track and the best track had to be lower
than a given threshold, taken as \SI{500}{km}. An example of a matching derived track taken from
one ensemble member and best track is shown in Figure \ref{fig:katrina_individual_match_em7}.
% N.B. em7 is counting from 0.
This shows how the cumulative and mean distances between a derived track (from ensemble member 8)
and a best track are calculated. All matching tracks across all the ensemble members for each of the
different configurations for Hurricane Katrina are shown in Figure
\ref{fig:katrina_six_tracking_configs}. This shows a typical hurricane that makes landfall. Some
noticeable points from this figure are TODOWRITE.

\begin{figure}[hb!]
    \centering
    \includegraphics[width=\linewidth]{figures/katrina_individual_match_em7}
    \caption{Individual match between best track (red) and derived track (blue dashed) for Hurricane
        Katrina. The derived track was generated from the \SI{850}{hPa} pressure level using the
        original resolution and ensemble member 8. The match between the best track and the derived
        track is shown using yellow lines, which show the distance between the two tracks at each
        timestep. The cumulative distance is the sum of these distances, and the mean distance is
        the cumulative distance divided by the number of matches.
    }
    \label{fig:katrina_individual_match_em7}
\end{figure}

\begin{figure}[hbp]
    \centering
    \includegraphics[width=4.5in]{figures/katrina_six_tracking_configs}
    \caption{The six different tracking configurations tested for their fit with the best track of
        Hurricane Katrina. Each of the individual figures shows all of the tracks across the
        different ensemble members that were matched with Hurricane Katrina. Katrina started as a
        tropical depression to the East of Florida, and then strengthened over the Gulf of Mexico,
        where it became a hurricane. This section of the best track is particularly well matched in all
        of the figures, indicating that it is easier to track hurricanes and stronger storms in general due
        to their more pronounced natures. The tighter grouping of the higher resolution derived tracks
        around the best track can also be seen, indicating that they may be providing a better track. }
    \label{fig:katrina_six_tracking_configs}
\end{figure}

To judge the tracking ability of the different tracking configurations, this matching across all
best tracks and derived tracks was done for each year in the ten year period 2000-2009. For each
year, the cumulative distance between each derived track and best track across all ensemble members
was calculated. This number was divided by the total number of overlaps between derived tracks and
best tracks, so as a particular configuration that achieved relatively few matches would be
penalised. This produces a metric by which the various configurations can be judged, and each of
the configurations was then compared the others to see how they performed on this metric. The
results of this comparison can be seen in Figure \ref{fig:tracking_wins_losses}. It is clear that
higher resolution, downscaled data produces better tracks that the unscaled data. This suggests that
for tracking, scaling of three times the original resolution should be used.

What is less clear is which is better: the NSPL or the \SI{850}{hPa} pressure levels. It is hard to
pick between them based on performance grounds on this metric alone. However, theoretical reasons
for preferring the \SI{850}{hPa} pressure level were given in Section TODOREF. Therefore the
\SI{850}{hPa} level was chosen over the NSPL. This has the added benefit of making comparison
between this study and other studies easer, as almost all other studies use the \SI{850}{hPa}
pressure level, thereby allowing e.g. comparison of the vorticity values found.

\begin{figure}[hbp]
    \centering
    \includegraphics[width=\linewidth]{figures/tracking_wins_losses}
    \caption{Comparison for the number of wins/losses for each tracking configuration within the
        NSPL (top), the \SI{850}{hPa} pressure level (middle) and then the scale 3 groups. In both
        of the pressure level comparisons it is clear that the scale 3 configuration comes out on top,
        with eight wins in both cases. The comparison between scale 3 for the two pressure levels is less
        clear, with each configuration winning five times apiece.}
    \label{fig:tracking_wins_losses}
\end{figure}

\newpage
\section{Extra Field Collection}
\label{sec:extra_field_collection}

Up until this point in the processing, only the $u$ and $v$ wind fields at different pressure levels
have been utilised. From these, the vorticity has been calculated and tracks have been derived from
the maxima of these fields. However, to successfully classify part of a track as being of hurricane
strength, it is necessary to collect more fields. These fields will help to distinguish hurricanes
from non-hurricanes. To this end, the latitude, longitude, date and ensemble member were used to
obtain further variables from the 20CR dataset. These variables were:

\begin{enumerate}
   \item \textbf{Vorticity:} the value of the vorticity at the local maximum.
    \item \textbf{Minimum pressure:} the pressure minimum within \SI{1000}{km} of the vorticity
        maximum.
    \item \textbf{Distance to minimum pressure:} the distance from the vorticity maximum to the
        pressure minimum.
    \item \textbf{Ambient pressure difference:} the difference in pressure between the minimum
        pressure and the local mean pressure, as calculated by taking the surrounding 121 grid cells
        and averaging their pressures.
    \item \textbf{Temperature at NSPL:} the temperature at the height of the NSPL.
    \item \textbf{Temperature at \SI{850}{hPa}:} the temperature at the height of the \SI{850}{hPa}
        pressure level.
    \item \textbf{Temperature difference:} the temperature difference between the NSPL and
        \SI{850}{hPa} pressure levels.
    \item \textbf{Maximum wind speed:} the maximum wind speed in the nearest 121 grid cells as taken
        from the \SI{850}{hPa} pressure level.
    \item \textbf{Distance to maximum wind speed:} the distance (in km) from the vorticity
        maximum to the wind speed maximum.
    \item \textbf{Direction to maximum wind speed:} the angle (in radians) from the vorticity
        maximum to the wind speed maximum.
    \item \textbf{Convective available potential energy:} a measure of atmospheric instability.
    \item \textbf{Atmospheric water vaopur content:} amount of water vapour held by the atmosphere.
    \item \textbf{Relative humidity at the NSPL:} Initial experimentation showed that this field was
        not useful for hurricane detection so it was not collected for all years.
\end{enumerate}

% Figure \ref{TODOFIGREF} shows the pressure and vorticity surrounding the vorticity maxima for the
% Hurricane Katrina track (taken from ensemble member 8). Figure \ref{TODOFIGREF} shows the different
% fields plotted for each timestep that Katrina was tracked for.

Figure \ref{fig:katrina_best_derived_comparison} shows the derived pressure and maximum wind speed
for Katrina, plotted with the equivalent variables for the matching best track. From this figure, it
is clear that there is far more variability in both pressure and maximum wind speed for the best
track than there is for the derived track. This is to be expected, as the 20CR data is at a
resolution that will reduce the relative values of e.g. the maximum wind speed, as discussed in
\textcite{walsh2007objectively}. However, to more fully test this, the values of all pressures and maximum wind
speeds for all ensemble members was plotted against the corresponding best track pressures and
maximum wind speeds as is shown in Figure \ref{fig:press_max_ws_corr_2005}.

% TODOFIG: one timestep per day showing wind/vorticity for Katrina.
% TODOFIG: plot of all collected fields for Katrina.

\begin{figure}[hb!]
    \centering
    \includegraphics[width=\textwidth]{figures/katrina_best_derived_comparison}
    \vspace{-10pt}
    \caption{Time series graphs taken from Hurricane Katrina (2005), showing pressure and maximum
            wind speed taken from the best track for Katrina and the corresponding pressure from derived
            track (taken form ensemble member 8). }
    \label{fig:katrina_best_derived_comparison}
    \vspace{-10pt}
\end{figure}

\begin{figure}[ht!]
    \centering
    \includegraphics[width=\textwidth]{figures/press_max_ws_corr_2005}
    \vspace{-10pt}
    \caption{Correlation of best track and derived track pressures and maximum wind speeds. The
        $r^2$ values indicate that there is some correlation between these variables. The best track
        pressure is shown to vary far more than the derived track pressure, exhibiting a much larger lowest
        possible value. The best track maximum wind speed is also shown to vary more, exhibiting a much
        larger maximum wind speed. }
    \label{fig:press_max_ws_corr_2005}
\end{figure}

\newpage
\section{Classification}
\label{sec:classification}

% Description of the goal of classification and how it applies in this setting. Set out terminology
% that I am going to use.
% standard by which the various classifiers can be trained/judged.
Now that for each ensemble member a number of tracks have been derived from the 20CR data, and that
extra fields have been collected using the tracks' positions at different timesteps, it is possible
to use this information to classify each point of each track as either being a hurricane or not.
These points will be knows as cyclone data points (CDPs) and for a typical year there are around
100000 of these points, spread across the 56 ensemble members.

Classification is the process of taking a dataset and splitting it up into various categories based
on the attributes of each item in the dataset. For the purposes of hurricane classification, this is
a binary classification, i.e. a CDP is either a hurricane or it is not. More fine-grained approaches
are possible, such as splitting each CDP into a category from the Saffir-Simpson hurricane scale
\parencite{simpson1974hurricane}, but in this analysis the only concern was trying to correctly categorise whether or
not a given CDP was a hurricane. In classification terminology, each CDP represents one sample, and
the different fields for each CDP are known as features for that sample.

Given that each derived track has also been matched to a best track, it is also possible to use the
information from the best tracks dataset about whether the point on the best track is a hurricane or
not as an objective standard for whether or not the corresponding point on the derived track
represents a hurricane. This is represented visually in Figure \ref{fig:cdp_2005_with_hurrs}, which
shows every CDP for the 2005, as well as whether the individual point has been matched to a
hurricane in the best tracks dataset or not. In classification terminology, this matching represents
the gold standard by which the classification algorithms can be trained, and by which their
performance can be validated. This allows classification algorithms to take the information derived
from the 20CR and perform classification based on this information. Matching of features from a set
of samples, against a gold standard, is known as supervised machine learning. This is contrast to
other types of machine learning where there is no objective gold standard by which to train the
algorithms, known as unsupervised machine learning. An example of unsupervised machine learning can
be seen in \textcite{studholme2014objective}, where a K-means clustering algorithm is used to
determine the extratropical transition of cyclones.

\begin{figure}[hb!]
    \centering
    \includegraphics[width=\textwidth]{figures/cdp_2005_with_hurrs}
    \vspace{-10pt}
    \caption{Every CDP from all ensemble members for the year 2005, as well as indication of whether
        the CDP was matched to a hurricane (red circle) or not (blue cross). The difference between
        the top row (Pressure - Vorticity) and the bottom row (Temperature at \SI{850}{hPa} -
        Vorticity) shows that some hurricanes may be distinguishable from non-hurricanes when
        looking at one variable, but not another. For both rows, the middle and right figures show the
        overlap between the two sets.}
    \label{fig:cdp_2005_with_hurrs}
\end{figure}

\subsubsection{Classification Successes and Errors}

% Explanation of FPs and FNs, TPs and TNs. Talk about Gold standard and how I am using IBTrACS as a
% gold standard. Relationship to Type I/II errors (maybe). Venn diagram showing FP/FN/TP/TN.

When classifying data with a binary classifier for which a gold standard exists, there are four
possible results of an individual classification. These are:

\begin{enumerate}
    \item \textbf{True Positive (TP):} The classification predicts that the CDP is a hurricane and
        this is verified by the best tracks information.
    \item \textbf{True Negative (TN):} The classification predicts that the CDP is not a hurricane
        and this is verified by the best tracks information.
    \item \textbf{False Positive (FP):} The classification predicts that the CDP is a hurricane and
        this is refuted by the best tracks information.
    \item \textbf{False Negative (FN):} The classification predicts that the CDP is not a hurricane
        and this is refuted by the best tracks information.
\end{enumerate}

When classifying a dataset, the collection of these results can be represented visually with a Venn
diagram, as shown in Figure \ref{fig:tf_np_venn}. They can also be represented in tabular form in an
Error Matrix (also called a Contingency Table or Confusion Matrix). An example of an Error Matrix
can be seen in Table \ref{tab:tf_np_table}, which shows the successes and failures of a simple
classifier for a subset of the data.

\begin{figure}[hb!]
    \centering
    \includegraphics[width=0.6\textwidth]{figures/tf_np_venn_cropped}
    \vspace{-10pt}
    \caption{Venn diagram showing relationship of True Positives, False Positives, True Negatives and
        False Negatives. Everything inside the thick dashed line is objectively true, as determined
        by the gold standard. Everything inside the thin dotted line is predicted to be true, by the
        classification algorithm. The intersection of these two sets represents all True Positives.}
    \label{fig:tf_np_venn}
    \vspace{-10pt}
\end{figure}

\begin{table}[hb!]
    \centering
    \begin{tabular}{ l | c c }
                                & Condition Positive & Condition Negative \\
        \hline
        Classification Positive & \SI{40587}{} & \SI{14344}{} \\
        Classification Negative & \SI{23125}{} & \SI{2042954}{} \\
    \end{tabular}
    \caption{Shows the number of True Positives, False Positives, False Negatives and True Negatives
        for the calibration run with the Stochastic Gradient Descent classifier described later in this
        section. }
    \label{tab:tf_np_table}
\end{table}

\newpage
\subsubsection{Sensitivity, Positive Predictive Value and False Positive Rate}

From the success and error results, various metrics are defined that give a measure of how well a
particular classifier performs at various tasks. Of these metrics, three will be utilised in this
project:

\begin{enumerate}
    \item \textbf{Sensitivity:} This is defined as $\frac{TP}{TP + FN}$. In this context, it gives a
        measure of how many hurricanes are correctly detected from all the actual hurricanes. 1
        represents a perfect sensitivity.
    \item \textbf{Positive Predictive Value (PPV):} This is defined as $\frac{TP}{TP + FP}$. In this
        context, it gives a measure of how many of the predicted hurricanes from the classification
        algorithm are actually hurricanes. 1 represents a perfect PPV.
    \item \textbf{False Positive Rate (FPR):} This is defined as $\frac{FP}{FP + TN}$. In this
        context, it gives a measure of the detection rate of all actual hurricanes. 0 represents a
        perfect FPR.
\end{enumerate}

Sensitivity is often plotted against FPR to make a Receiver Operator Characteristic plot (ROC plot).
ROC plots are useful for determining which particular classifier performs best over a given training
and validation dataset. The top left corner of this plot represents perfect classification. Another
plot which will be used in this project is a plot of Sensitivity against PPV, which will again
be used to determine which of the classifiers performs best. This plot is more useful in the context
of this project, as for the given datasets, very high values of FP detection are seen, which means
that the FPR will typically be very low, and less information can be gleaned from this particular
metric. Also, as will be explained in the next section, the Sensitivity and the PPV can be used to
obtain an estimate of the number of actual hurricanes. Examples of these plots can be seen in
Figure \ref{fig:ppv_and_fpr_vs_sens}.

For the values in Table \ref{tab:tf_np_table} the values would be:

\begin{table}[hb!]
    \centering
    \begin{tabular}{ l l }
        Sensitivity: & \SI{0.637}{} \\
        PPV: & \SI{0.739}{} \\
        FPR: & \SI{6.97e-3}{} \\
    \end{tabular}
    \label{tab:success_metric_table}
\end{table}

\subsubsection{Estimate of the Number of Actual Hurricanes}
\label{sec:estimated_hurricanes}

When any of the classifiers is run on a dataset, it will produce a prediction of which of the
members of that dataset are hurricanes, $H_{predicted}$. The PPV value derived for that particular classifier will
allow an estimate of the number of TPs for that prediction. Likewise, from the estimated number of
TPs, the sensitivity will allow a means of estimating the number of actual hurricanes, $H_{estimated}$,
in the dataset. The equation for this is:

\begin{equation}
    H_{estimated} = \frac{PPV \times H_{predicted}}{sensitivity}
    \label{eqn:n_actual_hurricane}
\end{equation}

\subsection{Classifiers}

\subsubsection{Threshold Classifier}
% How this is related to a Decision Tree classifier, how it was optimised.
The threshold classifier is the simplest of the classifiers considered. It works by using different
values for thresholds for one or more of the collected fields for each CDP. Any CDPs which have
field value which are e.g. higher than the specified threshold are considered hurricanes, and any
which are lower or equal to the threshold are considered not hurricanes. Despite its simplicity,
good results are possible with the threshold classifier by judicious choice of the thresholds for
each of the fields. These thresholds were first picked manually, then a simple brute force search
was carried out for values on either side of the chosen values. The metric for success was the sum
of the sensitivity and the PPV for the given classifier, where a larger value is better (a value of
2 would be perfect for this metric). It is also very easy to configure this classifier so as to
increase either the sensitivity or PPV, although there is a trade-off between these two metrics, as
shown in Figure \ref{fig:threshold_sens_ppv_vort}. The optimum values for these thresholds are shown
in the following table:

\begin{table}[hb!]
    \centering
    \begin{tabular}{ l l }
        Vorticity: & \SI{10.4e-5}{s^{-1}} \\
        Maximum wind speed: & \SI{16.1}{ms^{-1}} \\
        Temperature at NSPL: & \SI{297.2}{K} \\
        Temperature at \SI{850}{hPa}: & \SI{286.7}{K} \\
        Ambient pressure difference: & \SI{563.4}{hPa} \\
    \end{tabular}
    \caption{Optimum threshold values.}
    \label{tab:threshold_values}
\end{table}

This classifier is a simple version of a decision tree classifier, which passes the incoming data
through a series of binary tests, determining which category a particular CDP should end up in based
in the results of these tests.

The approach of using thresholds is very similar to the approach taken by
\textcite{walsh1997objective}. The main difference between his approach and the one here is that in
this case the thresholds are applied after the processing has been done, thereby allowing far more
scope for experimenting with changes to these threshold values (i.e. the whole processing step does
not have to be run to experiment with different threshold values). This allows for a much faster
retrieval of optimum thresholds with regard to a given metric, although it comes at a cost of having
to run the data analysis in a way that gathers fields for all of the CDPs without being able to
discard fields based on whether or not they pass a particular threshold test.

\begin{figure}[ht!]
    \centering
    \includegraphics[width=\textwidth]{figures/threshold_sens_ppv_vort}
    \vspace{-10pt}
    \caption{Sensitivity and PPV are shown to vary with the vorticity threshold used for a threshold
        classifier (bottom). Their sum is shown in the top graph. For low vorticity threshold, high
        sensitivities are seen (because there are fewer FNs) and lower PPVs are seen (as there are
        many FPs). As the threshold increases, the sensitivity decreases and the PPV increases as
        the FNs increase and the FPs decrease. Their sum shows a clear maximum around the
        intersection of the two individual lines, at a vorticity value of \SI{1e-4}{s^{-1}}. All
        other threshold values are the same as in Table \ref{tab:threshold_values} }
    \label{fig:threshold_sens_ppv_vort}
\end{figure}

\subsubsection{Linear/Quadratic Discriminant Analysis Classifiers}

% How these work.
% TODOCHECK: I need to check quadratic discriminant analysis.
For binary classification, Linear Discriminant Analysis (LDA) and Quadratic Discriminant Analysis
(QDA) are two closely related classification algorithms that allow computationally cheap
classification of data where a gold standard is available under trained supervision.
\textcite{mclachlan2004discriminant} gives a full account of both approaches, and a brief description of how they
work will be given here. The principal idea is to calculate the mean and covariance for all the
features of each group of data: $\mu_1$, $\sigma_1$ and $\mu_2$, $\sigma_2$. $\mu_1$ will lie at the
heart of the first group of features, and $\sigma_1$ will give a measure of the spread of this group
amongst each of the feature axes, and likewise for the second group. LDA then finds the dividing
surface between the two means (a plane in the same number of dimensions as there are features) based
on the relative values of the two variances. QDA is similar except that the dividing surface is
governed by a quadratic equation, and thus affords greater discriminating power between the two
groups.

These algorithms were not implemented from scratch; instead the implementations in the Python
library scikit-learn were used \parencite{scikitLearn2011}.

\subsubsection{Stochastic Gradient Descent Classifiers}
% How this works. Its relation to SVM.
Stochastic Gradient Descent (SGD) classifiers are an efficient class of classifiers that can scale
to 2 million samples, and are thus suitable for the problem in this project
\parencite{singer07pegasos, bottou2008tradeoffs}. This makes them more suitable than e.g. a Support
Vector Machine (SVM) classifier \parencite{cortes1995support}, to which they are closely related.
SVM classifiers typically only scale to around \SI{10e5}{} samples and therefore would not scale to
the number of samples used in this study. To understand the essentials of how an SGD classifier
works, it is useful to consider how an SVM classifier works first.

SVM classifiers work by calculating the hyperplane or set of hyperplanes which separate the two
categories into two distinct groups with the clearest possible gap between them. In the case where
there is no clean divide between the two categories, SVM classifiers will pick the hyperplanes which
most cleanly divide them \parencite{cortes1995support}. By use of the kernel trick
\parencite{aizerman1964theoretical}, SVM classifiers are not restricted to linear division between
the categories, but can represent more complex surfaces than just hyperplanes. The locations of the
hyperplanes which best separates the samples are calculated analytically.

SGD classifiers use the same underlying idea of hyperplanes to split the samples into two distinct
categories, however instead of calculating the location of the hyperplanes analytically, their
position is found iteratively using an SGD algorithm for optimising the position of this hyperplane.
This allows SGD classifiers to scale to far more samples than their SVM counterparts, whilst still
solving the same underlying problem (albeit possibly less optimally).

Again, the scikit-learn implementation for this classifier was used \parencite{scikitLearn2011}.
This particular classifier is sensitive to feature scaling, which is where a larger absolute range
for e.g. pressure than vorticity leads to the classification being unduly biased towards this
feature. Therefore the input features must be scaled before the classifier is trained with this
dataset, and likewise any data to be classified must be scaled using this same scaling as was used
for the training dataset. Dataset scaling was provided by the scikit-learn library.

\subsubsection{Combining Classifiers}
With some of the classifiers, it was clear from examination of the output that they performed well
apart from producing FPs in areas of e.g. low vorticity or low temperature at \SI{850}{hPa}.
Therefore it made sense to combine the output of e.g. an SGD classifier with a threshold classifier.
The result of this is to reduce the number of FPs, and increase the number of TNs by the same
amount. This has the effect of increasing the PPV and FPR, but has no effect on the sensitivity.

\subsection{Classifier Performance}

% TODOWRITE:
% Figures showing performance of various classifiers in terms of FP/FN/TP/TN for 3 variables plotted
% against vorticity for training set. Discussion of performance differences.

% Explain how 20 years worth of data were split into a training/validation dataset.
\subsubsection{Training Dataset}
\label{sec:training_dataset}
For the training dataset, all tracks from all ensemble members from the even years between 1990 and
2008 were taken. Given that each track is comprised of multiple (at least six) CDPs, this leads to a
large number of samples. On average, there were \SI{211906}{} samples per year, leading to a total
sample size of 2.12 million samples. This training dataset was used as a way of picking optimum
values for the threshold classifier manually (as defined by a maximum combined value of sensitivity
and PPV), and then using a brute force around the chosen values to verify that these values were
indeed the local maximum. The training dataset was also used to train each of the LDA, QDA and SGD
classifiers.

\subsubsection{Validation Dataset} 
The validation dataset was similarly made up of all the tracks from all ensemble members from the
odd years between 1991 and 2009. This ensured that the training and validation datasets were
independent, yet still spanned the almost the same time period. It had a similar number of samples
as the training dataset: 2.14 million samples.

\subsubsection{Performance Metrics}

From the performance metrics shown in Table \ref{tab:classifier_performance_metrics} and Figure
\ref{fig:ppv_and_fpr_vs_sens}, there is clearly some difference in performance between the different
classifiers. The LDC classifier performs least well on both metrics on sensitivity, and poorly on
PPV, therefore it is not deemed suitable for classification in this context. The QDC categoriser
performs particularly well on sensitivity, but is worst for PPV and FPR. This does not preclude its
use for classification; if it were important to not miss as many hurricanes as possible, and a high
number of FPs was acceptable, than it may be a suitable classifier. Indeed, coupled with a threshold
classifier (``QDC/Thresh.''), it performs well on both sensitivity and PPV metrics. The two
remaining classifiers, the threshold and the SGDC, perform similarly well on sensitivity, but
the SGDC classifier outperforms the threshold classifier on the PPV, therefore it is chosen as the
classifier to use.

\begin{figure}[hb!]
    \centering
    \includegraphics[width=\textwidth]{figures/ppv_and_fpr_vs_sens}
    \caption{Classifier performance shown for calibration (top) and validation (bottom) datasets,
        showing PPV vs sensitivity (left) and FPR vs sensitivity - Receiver Operating Characteristic
        plot (ROC plot, right). Note that x-axis on the ROC plots only goes from 0 to 0.1. }
    \label{fig:ppv_and_fpr_vs_sens}
\end{figure}

\begin{table}[hb!]
    \centering
    \begin{tabular}{ l | l l r | l l r | }
        \cline{2-7}
        & \multicolumn{3}{c |}{Training} & \multicolumn{3}{c |}{Validation} \\
        & \multicolumn{1}{c}{Sens.} & \multicolumn{1}{c}{PPV} & \multicolumn{1}{c|}{FPR} &
            \multicolumn{1}{c}{Sens.} & \multicolumn{1}{c}{PPV} & \multicolumn{1}{c|}{FPR} \\
        \hline
% cal: 0.6385453289804118, 0.702436244971252, 0.008377007122935034, 
% val: 0.6702334298858409,0.6811546520286066, 0.008826099523583321
        \multicolumn{1}{|c|}{Threshold} & 0.639 & 0.702 & \SI{8.38e-3}{} & 0.670 & 0.681 & \SI{8.82e-3}{} \\
% cal: 0.6177172275238574, 0.6129645204498022, 0.012078950156953441
% val: 0.6312659737604361, 0.6053164722412835, 0.011579432259338635
        \multicolumn{1}{|c|}{LDC} & 0.618 & 0.612 & \SI{12.08e-3}{} & 0.631 & 0.605 & \SI{11.58e-3}{} \\
% cal: 0.7484618282270216, 0.5963358969549177, 0.015689997268261573
% val: 0.7647299369568922, 0.5755725973992665, 0.015864258395292107
        \multicolumn{1}{|c|}{QDC} & 0.748 & 0.596 & \SI{15.69e-3}{} & 0.765 & 0.576 & \SI{15.86e-3}{} \\
% cal: 0.6370385484681065, 0.7388724035608308, 0.006972251953776264
% val: 0.6369398534673709, 0.7170643750479552, 0.007070274695750501
        \multicolumn{1}{|c|}{SGDC} & 0.637 & 0.739 & \SI{6.97e-3}{} & 0.640 & 0.717 & \SI{7.07e-3}{} \\
% cal: 0.6629363385233551, 0.6684761965054444, 0.010181801566909607 
% val: 0.6819219628556824, 0.6619035805838088, 0.009799161057981525
        \multicolumn{1}{|c|}{QDC/Thresh.} & 0.663 & 0.668 & \SI{10.18e-3}{} & 0.682 & 0.662 & \SI{9.79e-3}{} \\
        \hline
    \end{tabular}
    \caption{Performance metrics for each of the classifiers.}
    \label{tab:classifier_performance_metrics}
\end{table}

\chapter{20\ts{th} Century Analysis Results}
\label{chap:results_analysis}
% Data analysis and results.

% TODOWRITE: 4 eras of interest, where should it go? I think in the IBTrACS section for now.
\section{Hurricane Frequencies Over the 20\ts{th} Century}

Figure \ref{fig:20th_century_hurricane_timesteps} shows a comparison of the number of hurricane
timesteps (as defined in Section \ref{sec:ibtracs}) taken from the IBTrACS dataset, and calculated
from the 20CR dataset. To work out the estimate for the number of hurricanes, the detection
procedure was run over each year of the period from 1890 to 2009, and for each ensemble member. This
was used to work out a value for the predicted number of hurricanes for each year, with one value
per ensemble member. The sensitivity and PPV from the years 1990 to 2009 (the training and
validation period combined) were used in the formula given in Section
\ref{sec:estimated_hurricanes}, to give the estimated number of hurricanes for each year. The
maximum and minimum numbers of hurricanes in any ensemble member were used to plot the light blue
area, and the mean of the ensemble members was plotted using a blue dashed line.

From the top graph in Figure \ref{fig:20th_century_hurricane_timesteps}, there is clearly a marked
correlation between the two time series. There is also a suggestion of a discrepancy between the two
time series in the pre-Panama canal era and the aircraft reconnaissance era, with the estimated
hurricanes from the 20CR dataset being greater in the pre-Panama canal era, and fewer in the
aircraft reconnaissance era. This is borne out by the bottom graph in Figure 
\ref{fig:20th_century_hurricane_timesteps}, which shows the best tracks hurricanes minus the
estimated hurricanes. Here there is a clear sign that the best tracks number of hurricanes is
substantially lower than the estimated hurricanes in the pre-Panama canal era, and also that the
number of best tracks hurricanes is substantially larger in the aircraft (and early satellite) era.

\begin{figure}[ht!]
    \centering
    \includegraphics[width=\textwidth]{figures/20th_century_hurricane_timesteps}
    \caption{(top) The hurricane timesteps per year as taken from the IBTrACS best tracks database
        (red), and the mean across ensemble members of the estimated values taken from the 20CR
        dataset (blue dashed). The blue shading shows the difference between the maximum and maximum
        taken from the ensemble members, showing the range of values for each year. The division
        between four eras: pre-Panama Canal, Panama Canal, aircraft reconnaissance and satellite are
        shown with dashed black vertical lines. (bottom) The estimated value subtracted from the best
        tracks value for each year. A negative value means that the estimation is larger than the best
        tracks value.}
    \label{fig:20th_century_hurricane_timesteps}
\end{figure}

\newpage
Figure \ref{fig:20th_century_hurricane_corr} shows the correlation of the two time series, first
over the whole of the time period, and second by dividing the time period into two roughly equal
periods, based on the start of aircraft reconnaissance (1944). Given the high $r^2$ value of \SI{0.63}{}
shown in the first correlation plot, these series are very well correlated. The gradient is less than
unity though which implies that the estimated hurricanes is underestimating how many hurricanes there
were over the whole period (although this effect is slightly mitigated by the positive y-intercept).
From analysing the two time periods separately, as in the second correlation plot, a clear
distinction between the two periods can be seen, with the pre-1944 estimated numbers clearly
estimating more hurricanes than were seen in the best tracks database, and the reverse being true in
the post-1944 period. The pre-1944 period has a gradient of TODOCHECK and a y-intercept of TODOCHECK.
The post-1944 period has a slop of TODOCHECK ad a y-intercept of TODOCHECK.

\begin{figure}[hb!]
    \centering
    \includegraphics[width=\textwidth]{figures/20th_century_corr}
    \caption{Correlation plots showing the whole of the 20\ts{th} Century (left) and dividing the
    20\ts{th} Century into pre-1944 and post-1944 (right). From the left plot, the high value for
    $r^2$ indicates good correlation between the two datasets. From the right plot,  the different
    gradients of the lines of best fit indicate that in the pre-1944 period the estimated tracks
    were on average higher than the best tracks, and in the post-1944 period the estimates tracks were
    lower.}
    \label{fig:20th_century_hurricane_corr}
\end{figure}

\newpage
Figure \ref{fig:20th_century_hurricane_corr_split} shows a finer breakdown of the two series, based
on the four eras listed in Section TODOREF. This again shows that the number of estimated hurricanes
in the pre-Panama Canal era is far higher than in the best tracks database, both due to the large 
positive y-intercept and the gradient which is greater than unity. Reasons for this are discussed in
Section TODOREF. In the next era, the Panama era, the gradient is slightly below unity, and the
y-intercept is small and positive, which means that the two time series are well matched. The most
surprising era is the aircraft reconnaissance era, for which the gradient is substantially less than
unity, and the y-intercept is very close to zero. This means that during the aircraft reconnaissance
era, the estimated hurricanes were almost always lower than the best tracks hurricanes. Reasons as
to why this might be are discussed in Section TODOREF.  Finally, in the satellite era, the gradient is
very close to unity, and the y-intercept is significantly negative, indicating that during this era
the estimated number of hurricanes is systematically lower than the best tracks number of
hurricanes.

\begin{figure}[ht!]
    \centering
    \includegraphics[width=\textwidth]{figures/20th_century_corr_split}
    \caption{Individual correlation plots for each of the four eras. $r^2$ values were between 0.64
        and 0.80, representing good correlation between the two time series for each era. The high
        gradient for the pre-Panama era and the low gradient for the aircraft era are particularly noteworthy. }
    \label{fig:20th_century_hurricane_corr_split}
\end{figure}

\newpage
\subsection{Trends in Hurricane Frequencies}
\label{sec:trends_freq}

For the four eras in question, it is interesting to note whether there are any trends in the numbers
of hurricanes, both in the best tracks dataset, and in the estimated hurricane numbers. This is done
by fitting a straight line to the time series, and performing a two-sided test on the $p$ value of the
fitted line, as in \textcite{vecchi2008estimates, emanuel2010tropical}. The $p$ values for the
different eras are shown in Table \ref{tab:era_p_value}.

\begin{table}[ht!]
    \centering
    \begin{tabular}{ l | r r | r r | }
        \cline{2-5}
        & \multicolumn{2}{c |}{IBTrACS }& \multicolumn{2}{c |}{20CR} \\
        \hline
        \multicolumn{1}{| l |}{Era} & gradient & $p$ value & gradient & $p$ value \\
        \hline
        \multicolumn{1}{| l |}{pre-Panama Canal} & -4.93 & 0.018 & -5.41 & 0.013 \\
        \multicolumn{1}{| l |}{Panama Canal} & 0.41 & 0.768 & -0.81 & 0.513 \\
        \multicolumn{1}{| l |}{Aircraft} & 1.82 & 0.391 & 0.41 & 0.768 \\
        \multicolumn{1}{| l |}{Satellite} & 1.07 & 0.190 & 2.48 & 0.001 \\
        \hline
    \end{tabular}
    \caption{}
    \label{tab:era_p_value}
\end{table}

\newpage
The test for statistical significance employed in this study is to test to see if $p \leq 0.05$.
This test is used to see whether the null hypothesis that the time series in question is not
changing (i.e. its gradient is zero) can be rejected.  This is the same as was used in
\textcite{vecchi2008estimates}. Based on this test for significance, three of the time series over
the eras in Table \ref{tab:era_p_value} show statistically significant trends. The pre-Panama
Canal era (1890 - 1914) for hurricane frequencies for best track and estimated track numbers both
show significant downwards trends. However, the most significant change is shown by the estimated
hurricane numbers in the satellite era (1966 - 2009) from only the 20CR dataset, which has a $p$
value of \SI{0.001}{}. The same era in the best tracks number is not seen to show any significant
rise, although the gradient of the line is positive.

% \section{Hurricane Dissipated Power Index Over the 20\ts{th} Century}

\chapter{Discussion}
\label{chap:discussion}

\section{Tracking Algorithms}
\label{sec:discussion_tracking_algs}.

\chapter{Conclusions}
\label{chap:conclusion}

% Not included in word count.
\addcontentsline{toc}{chapter}{Auto-critique}
\chapter*{Auto-critique}

% Harvard style bibliography.
\addcontentsline{toc}{chapter}{References}
\printbibliography[title={References}]

%\appendix
%\section{Additional information}

\end{document}
